\documentclass[10pt,a4paper,oneside]{article}
\usepackage[latin1]{inputenc}
\usepackage{amsmath}
\usepackage{amsfonts}
\usepackage{amssymb}
\usepackage{graphicx}

\begin{document}	
\textbf {Explain} \\  \vspace{0.5cm}

Martingale theory $\sim$ fair game: kh�ng c� su ch�nh lech gi�, gi� giu nguy�n theo thoi gian, ph� hop voi ti�u chuan cua tat ca moi nguoi, kh�ng c� nguoi thang nguoi mat. \\ \vspace{0.5cm}

$\dfrac{dS_t}{S_t}$ is rate of return over $\Delta t$ voi ky vong $\mu$ (ky vong loi nhuan kh�c nhau t�y v�o tung investor trong khoang thoi gian $\Delta t$) v� su bien dong, sai lech giua c�c ky vong d� l� $\sigma$ trong buoc chuyen dong ngau nghi�n cua gi� co phieu theo Brownian.\\ \vspace{0.5cm}

$\mu$ bien thien theo Phuong trinh tuyen tinh, nghia la thoi gian c�ng d�i th� ky vong loi nhuan cua investor c�ng cao\\ \vspace{0.5cm}

$dW_t$ l� bien ngau nhien tu $t$ den $t+\Delta t$. $W_t$ Wiener process voi phan bo xac suat normal ~ N(0, $\Delta t$) n�n thoi gian c�ng d�i th� do bien dong c�ng lon \\ \vspace{0.5cm}

$B_t$ Risk-free asset:  government bond \\ \vspace{0.5cm}

Def:  Let $(\Omega;\mathcal{F})$ be a sample space. Two probability measures $\mathbb{P}$ and $\mathbb{Q}$ on $(\Omega;\mathcal{F})$
are said to be equivalent if
P(A) = 0 , Q(A) = 0
for all such A. \\ \vspace{0.5cm}

Equivalent martingale measure la cong cu toan hoc, giup giai quyet duoc quy ve 1 chuan muc nhat dinh de giup giai quyet cac van de trong te mot cach de dang va hieu qua hon,day la do do xac suat tuong duong voi do do xac suat trong the gioi thuc, tuong duong theo nghia xac suat, co nghia la "P(A) = 0 or 1, then Q(A) = 0 or 1". con cac gia tri khac bien doi tuong quan theo 1 he so $\theta$ nhat dinh". \\ \vspace{0.5cm}

Dua vao y tuong do, Girsanov da chung minh tim ra he so nay trong Girsanov's theorem.  
                                  
                     
\end{document}