\documentclass[12pt,a4paper,left=2cm,right=2cm,oneside,titlepage]{report}
\usepackage[utf8]{inputenc}
\usepackage{amsmath}
\usepackage{amsfonts}
\usepackage{amssymb}
\usepackage{cases}
\numberwithin{equation}{section}
\usepackage{mathtools}
\usepackage{commath}
\usepackage{makeidx}
\usepackage{graphicx}
\usepackage{fancyhdr}
\begin{document}
	\thispagestyle{empty}
	
	\begin{center}
		\fontsize{12pt}{18pt}\selectfont \textbf{VIETNAM NATIONAL UNIVERSITY - HO CHI MINH CITY\\
			INTERNATIONAL UNVERSITY\\
			DEPARTMENT OF MATHEMATICS} \vspace{0.8cm}
		
		\begin{figure}[htp]
			\begin{center}
				\includegraphics[scale=.6]{logo}
			\end{center}
		\end{figure}	
		\begin{center}
			\fontsize{12pt}{18pt}\selectfont \textbf{GRADUATION THESIS:} \\ \vspace{36pt}
			
			\fontsize{14pt}{16pt}\selectfont \textbf{PRICING EUROPEAN BARRIER OPTIONS WITH REBATES} \\ \vspace{2pt}
			
			\fontsize{12pt}{18pt}\selectfont \textbf{Submitted in partial fulfillment of the requirements for the degree of \\
				BACHELOR OF ENGINEERING in\\ 
				FINANCIAL ENGINEERING AND RISK MANAGEMENT} \\ \vspace{18pt}
			
			\fontsize{12pt}{18pt}\selectfont \textbf{Student's Name:	Ta Thi Phuong Dung\\
				Student's ID:	MAMAIU13067\\
				Thesis Supervisor:	Dr. Le Nhat Tan}
			
		\end{center}
	\end{center}
	
	\chapter*{Executive Summary}
	
	\fontsize{11pt}{20pt}\selectfont In Vietnam, derivatives market has just started officially very recently, in August, 2017, with
	futures contracts on VN30 index. It is a very new investment area for Vietnamese investors.
	This market is however expected to strongly develop soon and then enhance greatly Vietnamese
	economy. In fact, trading volume on futures contracts on VN30 index is increasing significantly over last few months, and is expected to increase with an even faster rate. One important type of financial derivatives products is options. In Vietnam, covered call (a type of call options)
	will be traded soon. Options will bring greater leverage to speculators and bring more risk management tools for hedgers. Options thus have great potential chance in Vietnamese financial markets. Understanding clearly the pricing formulas for these products is very urgent, then the
	object of the thesis formulate the pricing models of European barrier options with rebates using
	the probabilistic approach.\\[0.5cm]
	Mathematical models have been used in economics for a long time and these are very important for the financial markets. The scale of mathematical techniques in this context include deriving the Black - Scholes - Merton model by the Martingale approach which is model of European options. Then European down and out call options with rebates is formulated by using theory of Wiener process like Reflection Principle, First Passage Time, Markov Property and Stochastic Differential Equations as Ito Calculus and One-Dimensional Difusion Process. Further, testing real data for normal distribution using R-Studio software. Lastly, using the final formula to calculate the option price of stock price in Vietnam. The stock is applied that must be satisfied given conditions by the previous testing.	
	\chapter*{Thesis Outline}
	First of all, the history of derivative instruments and its popularity is given which leads to this thesis's urgentcy. Moreover, the overall of derivative instruments theory is introduced for classifying financial knowledges in this context, these is involved in \textbf{Chapter 1}.  Then, \textbf{Chapter 2} generalizes the formula and theory of mathematics as General Probability, Wiener Process, Stochastic Differential Equations and Change of Measure that is necessary.  \\[0.5cm]
	\textbf{Chapter 3} presents how to combine the financial knowledges with mathematical techneques for formulating the Black-Schole-Merton model which price the European option and applies this model into FPT stock. 	\textbf{Chapter 4} is similar to \textbf{Chapter 3}, but the model is created for pricing European Barrier Options with
	Rebates and FPT stock is also mentioned. \\[0.5cm]
	Finally, some conclusions including current situation of derivatives in Vietnam and main concept within this topic
	area will be stated in \textbf{Chapter 5}.
\end{document}